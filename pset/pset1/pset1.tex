\documentclass{article}
\usepackage[utf8]{inputenc}

\usepackage{DefaultPackages, GeneralCommands, MathematicA}



\renewcommand{\thesubsection}{\thesection.\alph{subsection}}
 
  
\begin{document}
\section{}
\section{}
\section{}
\section{}
\subsection{}
Consider a single perceptron. Let $\sigma$ be the activation function of the perceptron i.e. $\sigma(x)= \one (x >0)$ . Let $w$ denote the weights and $b$ the bias. Then the output of the perceptron for an input $x$ is $\sigma(w x +b)$. Rescaling the weights and bias by $c>0$ is 

$$\sigma(cw x +cb)=\sigma(c(w x +b))= \one (c(w x +b) >0)= \one (w x +b >0)=\sigma(cw x +cb).$$

We used $c>0$. Since this holds true for every perceptron in a perceptron network, rescaling does not behave the behaviour. 
\subsection{}
The sigmoid function is
$$\sigma(x)=\frac{1}{1+e^{-x}}.$$

Then 
$$\sigma(c(w x +b))=\frac{1}{1+e^{-c(w x +b)}}=\frac{1}{1+(e^{-(w x +b)})^c}.$$

We see that for $w x +b\neq 0$ we have $\lim_{c\to \infty}\sigma(c(w x +b))=\one (w x +b >0)$, which is exactly the behavior of a perceptron. For $w x +b\neq 0$ we have $\sigma(c(w x +b))=0.5$ for all $c$.
\subsection{}
$W_1 = ()$
\end{document}